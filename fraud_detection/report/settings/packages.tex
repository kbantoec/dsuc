%% packages

%---------------------------------------------------------------------
% FONT SET UPS
%---------------------------------------------------------------------

\usepackage[T1]{fontenc}
\usepackage[utf8]{inputenc}
\usepackage{lmodern}
\usepackage[super]{nth}
%\usepackage{fullpage} % no needed with geometry package
\usepackage{amsmath} % needed for command eqref
\usepackage{amsfonts} % needed for math fonts
\usepackage{amssymb}
\usepackage{xcolor} % required for specifying colors by name
\usepackage{color, colortbl} % colortbl to colour tables with \cellcolor{} \rowcolor{}, etc.
\usepackage[]{hyphenat} % to use \allowbreak

\definecolor{dcblue}{RGB}{59,170,202} % data camp blue
\definecolor{dclblue}{HTML}{eaf4f7} % light data camp blue 
\definecolor{dcdblue}{HTML}{2c87a0} % dark data camp blue
\definecolor{impactblue}{HTML}{007acc}

%---------------------------------------------------------------------
% ALGORITHMS
%---------------------------------------------------------------------
\usepackage[ruled,vlined]{algorithm2e}
\usepackage{algpseudocode}

%---------------------------------------------------------------------
% LISTS
%---------------------------------------------------------------------

\usepackage{scrextend}
\addtokomafont{labelinglabel}{\sffamily}
\usepackage[ampersand]{easylist}
\usepackage{enumitem}  % to use [noitemsep]

%---------------------------------------------------------------------
% HYPERLINK SET UPS
%---------------------------------------------------------------------

\usepackage[
	colorlinks = true,
	%,breaklinks
	]{hyperref} % needed for creating hyperlinks, the option colorlinks=true gets rid of the awful color boxes, breaklinks breaks long links (list of figures).
	
\hypersetup{% options in comments --> already been used --> (hyperref) setting option has no effect
	hidelinks,
	%backref		= true,
	%pagebackref	= true,
	%hyperindex	= true,
	colorlinks		= false,
	citecolor 		= {dcblue},
	breaklinks		= true,
	urlcolor		= {dcblue},
	%bookmarks		= true,
	%bookmarksopen	= false,
	pdftitle		= {Title},
	pdfauthor 		= {Author}
}
\usepackage[hyphenbreaks]{breakurl} % benötigt für das Brechen von URLs in Literaturreferenzen, hyphenbreaks auch bei links, die über eine Seite gehen (mit hyphenation).

%---------------------------------------------------------------------
% FIGURE SET UPS
%---------------------------------------------------------------------

\usepackage{graphicx}\graphicspath{{figures/}} % needed for \includegraphics

%---------------------------------------------------------------------
% AUTOMATIC PARAGRAPHS
%---------------------------------------------------------------------

\usepackage{blindtext} % needed for creating dummy text passages
\usepackage{lipsum} % inserts dummy text

%---------------------------------------------------------------------
% TABLE SETTINGS
%---------------------------------------------------------------------

\usepackage{array} % needed for creating tables
\usepackage{rotating, booktabs} % required for nicer horizontal rules in tables
\usepackage{tabularx} % sets columns-size according to the content
\usepackage{longtable} % needed for long tables over pages
\usepackage{caption} % to use caption*{} to insert legends
\renewcommand{\captionfont}{\small\sffamily}
\renewcommand{\captionlabelfont}{\sffamily}
\setlength{\captionmargin}{70pt}
%\usepackage{tabu}
\usepackage{multirow}

%---------------------------------------------------------------------
% DRAWING SETTINGS
%---------------------------------------------------------------------

\usepackage{tikz} % required for drawing custom shapes --> https://fr.overleaf.com/learn/latex/TikZ_package

%---------------------------------------------------------------------
%	BIBLIOGRAPHY AND INDEX
%---------------------------------------------------------------------

% Good video:
% https://www.youtube.com/watch?v=KS9GvK7cvmo

%\usepackage[style=alphabetic,
%citestyle=numeric,
%sorting=nyt,
%sortcites=true,
%autopunct=true,
%babel=hyphen,
%hyperref=true,
%abbreviate=false,
%backref=true,
%backend=biber]{biblatex}
%\addbibresource{bibliography.bib} % BibTeX bibliography file
%\defbibheading{bibempty}{}
%
%\usepackage{calc} % For simpler calculation - used for spacing the index letter headings correctly

%\usepackage[round,authoryear]{natbib} % needed for cite and abbrvnat bibliography style
%\usepackage{natbib}  % \usepackage[options]{natbib}

\usepackage{cite}  % to use commands such as \citeA
\usepackage{apacite}
\usepackage{makeidx} % required to make an index
\makeindex % tells LaTeX to create the files required for indexing

%---------------------------------------------------------------------
%	GENERATING LISTS OF CUSTOM ENVIRONMENT
%---------------------------------------------------------------------

% List of functions

%\usepackage{tocloft} % The package lets you customise your table of contents, list of figures or list of tables

% /!\ Problem with this package versus titlesec! 

% Good explanations of how to use the package
% https://texblog.org/2008/07/13/define-your-own-list-of/
% cf documentation package

\usepackage{amsthm} % needed for new theorems
\usepackage{thmtools} % needed for \listoftheorem

% Editing Format of List of Theorems in thmtools:
% https://tex.stackexchange.com/questions/180747/editing-format-of-list-of-theorems-in-thmtools

% Generating lists of custom environment:
% https://tex.stackexchange.com/questions/16494/generating-lists-of-custom-environment

\renewcommand{\listtheoremname}{List of Functions}

%The \makeatletter command temporarily defines »@« as a normal character to enable changes to internal LaTeX macros outside packages (STY) or classes (CLS). With \makeatother this process is reversed and the »@« is set to its original character category (other). The »@« is used to protect the internal LaTeX macros. Hence you should be very careful when using these two commands. cf. https://latex.org/forum/viewtopic.php?t=3732

%\makeatletter
%\declaretheoremstyle[
%	spaceabove 	= 5pt, 
%	spacebelow		= 5pt,
%	headfont		= \Huge\itshape,
%	notefont		= \normalfont\sffamily, 
%	notebraces		= {(}{)},
%	bodyfont		= \normalfont\sffamily,
%	postheadspace	= 0.5em,
%	qed				= {},
%	postheadhook	= {%
%  		\ifx\@empty\thmt@shortoptarg
%    	\renewcommand\addcontentsline[3]{}
%  		\fi}%
%	]{theorem}
%\makeatother
%
%\declaretheorem[style=theorem]{theorem}
%
%\makeatletter
%\def\ll@theorem{%
%  \protect\numberline{\csname the\thmt@envname\endcsname}%
%  \ifx\@empty\thmt@shortoptarg
%    \thmt@thmname
%  \else
%    \thmt@shortoptarg
%  \fi}
%\def\l@thmt@theorem{} 
%\makeatother

%---------------------------------------------------------------------
%	MAIN TABLE OF CONTENTS
%---------------------------------------------------------------------

\usepackage{titletoc} % required for manipulating the table of contents

\contentsmargin{0cm} % removes the default margin

%\titlecontents{section}[left]
%{above-code}
%{numbered-entry-format}
%{numberless-entry-format}
%{filler-page-format}
%[below-code]

%When defining entries, use \addvspace if you want to add vertical space, and \\* instead of \\ for line breaks.

% Part text styling
% https://tex.stackexchange.com/questions/180112/how-to-modify-titlecontentspart

\titlecontents{part}[0cm]
{\addvspace{20pt}\centering\Large\itshape}
{}
{Part \contentslabel[\thecontentslabel]{1.25cm}}
{}

% Chapter text styling

\titlecontents{chapter}[1.25cm] % Indentation
{\addvspace{15pt}\large\sffamily\bfseries} % Spacing and font options for chapters
{\color{dcblue!60}\contentslabel[\Large\thecontentslabel]{1.25cm}\color{dcblue}} % Chapter number
{\color{dcblue!60}}  
{\color{dcblue!60}\normalsize\sffamily\bfseries\;\titlerule*[.5pc]{.}\;\thecontentspage} % Page number

% Section text styling

\titlecontents{section}[1.25cm] % Indentation
{\addvspace{5pt}\sffamily\bfseries} % Spacing and font options for sections
{\contentslabel[\thecontentslabel]{1.25cm}} % Section number
{}
{\sffamily\hfill\color{black}\thecontentspage} % Page number
[]

% Subsection text styling

\titlecontents{subsection}[1.25cm] % Indentation
{\addvspace{1pt}\sffamily\small} % Spacing and font options for subsections
{\contentslabel[\thecontentslabel]{1.25cm}} % Subsection number
{}
{\sffamily\;\titlerule*[.5pc]{.}\;\thecontentspage} % Page number
[] 

% List of figures

\titlecontents{figure}[0em]
{\addvspace{-5pt}\sffamily}
{\thecontentslabel\hspace*{1em}}
{}
{\ \titlerule*[.5pc]{.}\;\thecontentspage}
[]

% List of tables

\titlecontents{table}[0em]
{\addvspace{-5pt}\sffamily}
{\thecontentslabel\hspace*{1em}}
{}
{\ \titlerule*[.5pc]{.}\;\thecontentspage}
[]

% List of theorems

\titlecontents{theorem}[0em]
	{\addvspace{-5pt}\sffamily}
	{\sffamily\thecontentslabel}
	{\hspace*{1em}}
	{\ \titlerule*[.5pc]{.}\;\thecontentspage}
	[]

%---------------------------------------------------------------------
% LISTINGS
%---------------------------------------------------------------------

\usepackage{listings} % needed to add non-formatted text as you would do with \begin{verbatim}

\lstloadlanguages{Python} % needed to recognize Python programming language

\usepackage{upquote} % to set upquote equal true in lstset
\usepackage{listingsutf8}

\definecolor{codegreen}{rgb}{0,0.6,0}
\definecolor{codegray}{HTML}{405f8f}
\definecolor{codepurple}{rgb}{0.58,0,0.82}
\definecolor{backcolour}{rgb}{0.95,0.95,0.92}

\lstdefinestyle{hispythonstyle}{
    language = Python,
    backgroundcolor=\color{backcolour},   
    commentstyle=\color{codegreen},
    keywordstyle=\color{magenta},
    numberstyle=\tiny\color{codegray},
    stringstyle=\color{codepurple},
    basicstyle=\ttfamily\footnotesize,
    breakatwhitespace=false,         
    breaklines=true,                 
    captionpos=b,                    
    keepspaces=true,                 
    numbers=left,                    
    numbersep=5pt,                  
    showspaces=false,                
    showstringspaces=false,
    showtabs=false,                  
    tabsize=2
}

\definecolor{azraq blue}{HTML}{4a69bd}
\definecolor{livid grey}{HTML}{919191}
\definecolor{very berry}{HTML}{B53471}
\definecolor{bara red}{HTML}{ED4C67}
\definecolor{sasquatch socks}{HTML}{FC427B}
\definecolor{light purple}{HTML}{c56cf0}
\definecolor{baltic sea}{HTML}{3d3d3d}

\lstdefinestyle{mypythonstyle}{
    literate = 
      {á}{{\'a}}1 {é}{{\'e}}1 {í}{{\'i}}1 {ó}{{\'o}}1 {ú}{{\'u}}1
      {Á}{{\'A}}1 {É}{{\'E}}1 {Í}{{\'I}}1 {Ó}{{\'O}}1 {Ú}{{\'U}}1
      {à}{{\`a}}1 {è}{{\`e}}1 {ì}{{\`i}}1 {ò}{{\`o}}1 {ù}{{\`u}}1
      {À}{{\`A}}1 {È}{{\'E}}1 {Ì}{{\`I}}1 {Ò}{{\`O}}1 {Ù}{{\`U}}1
      {ä}{{\"a}}1 {ë}{{\"e}}1 {ï}{{\"i}}1 {ö}{{\"o}}1 {ü}{{\"u}}1
      {Ä}{{\"A}}1 {Ë}{{\"E}}1 {Ï}{{\"I}}1 {Ö}{{\"O}}1 {Ü}{{\"U}}1
      {â}{{\^a}}1 {ê}{{\^e}}1 {î}{{\^i}}1 {ô}{{\^o}}1 {û}{{\^u}}1
      {Â}{{\^A}}1 {Ê}{{\^E}}1 {Î}{{\^I}}1 {Ô}{{\^O}}1 {Û}{{\^U}}1
      {Ã}{{\~A}}1 {ã}{{\~a}}1 {Õ}{{\~O}}1 {õ}{{\~o}}1
      {œ}{{\oe}}1 {Œ}{{\OE}}1 {æ}{{\ae}}1 {Æ}{{\AE}}1 {ß}{{\ss}}1
      {ű}{{\H{u}}}1 {Ű}{{\H{U}}}1 {ő}{{\H{o}}}1 {Ő}{{\H{O}}}1
      {ç}{{\c c}}1 {Ç}{{\c C}}1 {ø}{{\o}}1 {å}{{\r a}}1 {Å}{{\r A}}1
      {€}{{\euro}}1 {£}{{\pounds}}1 {«}{{\guillemotleft}}1
      {»}{{\guillemotright}}1 {ñ}{{\~n}}1 {Ñ}{{\~N}}1 {¿}{{?`}}1,
    language = Python,
    basicstyle = \ttfamily\footnotesize,
    breakatwhitespace = false, % sets if automatic breaks should only happen at whitespace
    breaklines = true, % sets automatic line breaking
    captionpos = b, % sets the caption-position to bottom
    numbers = left, % where to put the line-numbers; possible values are (none, left, right)
    firstnumber = 1, % Line numbers start with line 1
    numberstyle = \ttfamily\color{baltic sea}, %\ttfamily\color{azraq blue},
    %frame = single, % adds a frame around the code
    frame = leftline,
    framerule = 1pt,
    frameround = tttt,
    %framerule = \thicklines % or thinlines when rounded frame
    rulecolor = \color{baltic sea},
    otherkeywords = {self, None, NotImplementedError, assert}, % Add keywords here
    keywordstyle = \bfseries\color{very berry},
    commentstyle = \color{livid grey}, % comment style
    stringstyle = \color{azraq blue}, % string literal style
    emph = {__str__, __init__, __name__, __repr__, __eq__, __gt__, __ge__, __class__}, % Custom highlighting
    emphstyle = \color{sasquatch socks}, % Custom highlighting style
    aboveskip = 0.3cm, % Space between listings env. and surrounding text
    belowskip = 0cm,
    upquote = true, % makes all verbatim quotes single quotes
    showstringspaces = false, % to get rid of the "squat-u" between string spaces
    % backgroundcolor = \color[HTML]{eaf4f7}
}

\lstset{style=mypythonstyle}

% \lstset{
% inputencoding = utf8,
% % extendedchars=true,
% literate = {é}{{\'{e}}}1,
% language = Python,
% basicstyle = \ttfamily\footnotesize,
% breakatwhitespace = false, % sets if automatic breaks should only happen at whitespace
% breaklines = true, % sets automatic line breaking
% captionpos = b, % sets the caption-position to bottom
% numbers = left, % where to put the line-numbers; possible values are (none, left, right)
% firstnumber = 1, % Line numbers start with line 1
% numberstyle = \ttfamily,
% %frame = single, % adds a frame around the code
% frame = leftline,
% framerule = 2pt,
% frameround = tttt,
% %framerule = \thicklines % or thinlines when rounded frame
% rulecolor = \color[HTML]{3baaca},
% otherkeywords={self}, % Add keywords here
% keywordstyle = \bfseries\color[HTML]{3baaca}, % pink: \color[HTML]{ff0066}
% commentstyle = \color[HTML]{405f8f}, %\color{gray},% \color[HTML]{009973}, % comment style
% stringstyle = \color[HTML]{009432}, % \color[HTML]{6600ff}, % string literal style
% emph = {__str__,__init__, __name__, __repr__}, % Custom highlighting
% emphstyle = \color[HTML]{B53471}, % Custom highlighting style
% aboveskip = 0.3cm, % Space between listings env. and surrounding text
% belowskip = 0cm,
% upquote = true, % makes all verbatim quotes single quotes
% showstringspaces = false, % to get rid of the "squat-u" between string spaces
% % backgroundcolor = \color[HTML]{eaf4f7} % choose the background color; you must add \usepackage{color} or \usepackage{xcolor}; should come as last argument
% }

% Further information about frames:
% https://wiki.math.ntnu.no/_media/drift/stud/listings.pdf

%---------------------------------------------------------------------
% NEW THEOREM STYLES
%---------------------------------------------------------------------

%\usepackage{amsthm} % needed for new theorems

%\newtheoremstyle{stylename}% name of the style to be used
%  {spaceabove}% measure of space to leave above the theorem. 
%  {spacebelow}% measure of space to leave below the theorem. 
%  {bodyfont}% name of font to use in the body of the theorem
%  {indent}% measure of space to indent
%  {headfont}% name of head font
%  {headpunctuation}% punctuation between head and body
%  {headspace}% space after theorem head; " " = normal interword space
%  {headspec}% Manually specify head

\newtheoremstyle{examplestyle} % theorem style name
{5pt} % space above
{5pt} % space below
{\normalfont} % body font
{} % indent amount
{\bfseries\sffamily} % theorem head font
{\\[0.1cm]} % punctuation between head and body
{10mm} % space after theorem head
{\sffamily\thmname{#1}\nobreakspace\thmnumber{#2}\thmnote{\nobreakspace---\nobreakspace #3.}} % manually specify head

\newtheoremstyle{codestyle}
{5pt} 
{5pt} 
{\ttfamily} 
{} 
{\bfseries\sffamily} 
{\newline}
{10mm} 
{\sffamily\thmname{#1}\nobreakspace\thmnumber{#2}\thmnote{\nobreakspace---\nobreakspace {\bfseries\ttfamily #3}.}} % manually specify head

%\newtheoremstyle{blacknumbox} % Theorem style name
%{0pt}% Space above
%{0pt}% Space below
%{\normalfont}% Body font
%{}% Indent amount
%{\small\bf\sffamily}% Theorem head font
%{\;}% Punctuation after theorem head
%{0.25em}% Space after theorem head
%{\small\sffamily\thmname{#1}\nobreakspace\thmnumber{\@ifnotempty{#1}{}\@upn{#2}}% Theorem text (e.g. Theorem 2.1)
%\thmnote{\nobreakspace\the\thm@notefont\sffamily\bfseries---\nobreakspace#3.}\newline}% Optional theorem note

%---------------------------------------------------------------------
% NEW THEOREMS
%---------------------------------------------------------------------

% https://en.wikibooks.org/wiki/LaTeX/Theorems
% http://www.ctex.org/documents/packages/math/amsthdoc.pdf

% \theoremstyle --> plain, definition or remark.

\theoremstyle{examplestyle} % name of the style to be used
\newtheorem{exampleT}{Example} % new theorem
\newtheorem{definitionT}{Definition}
\newtheorem{excerptT}{Text excerpt}
\newtheorem{instructionT}{Instruction}

\theoremstyle{codestyle}
\newtheorem{functionT}{Function}


%\newcounter{dummy} 
%\numberwithin{dummy}{section}
%\theoremstyle{blacknumbox}
%\newtheorem{definitionT}{Definition}[section]

\theoremstyle{plain} % default
\newtheorem{thm}{Theorem}[section]
\newtheorem{lem}[thm]{Lemma}
\newtheorem{prop}[thm]{Proposition}
\newtheorem*{cor}{Corollary}

\theoremstyle{definition}
\newtheorem{conj}{Conjecture}[section]
% \newtheorem{defn}{Definition}[section]

\theoremstyle{remark}
\newtheorem*{rem}{Remark} % unnumbered theorem because of *
\newtheorem*{note}{Note} % unnumbered theorem because of *
\newtheorem{case}{Case}

%---------------------------------------------------------------------
%	DEFINITION OF COLORED BOXES
%---------------------------------------------------------------------

\RequirePackage[
	framemethod = default
	]{mdframed} % required for creating the theorem, definition, exercise and corollary boxes

% Definition box
\newmdenv[
	linewidth = 2pt,
	skipabove = 11pt,
	skipbelow = 0pt,
	rightline = false,
	leftline	= true,
	topline = false,
	bottomline = false,
	linecolor = dcblue,
	innerleftmargin = 5pt,
	innerrightmargin	= 5pt,
	innertopmargin = 0pt,
	leftmargin = 0cm,
	rightmargin = 0cm,
	innerbottommargin = 0pt
	]{dBox}

% Instruction box
\newmdenv[
	linewidth = 2pt,
	skipabove = 11pt,
	skipbelow = 0pt,
	rightline = false,
	leftline	= true,
	topline = false,
	bottomline = false,
	linecolor = dcblue,
	innerleftmargin = 5pt,
	innerrightmargin	= 5pt,
	innertopmargin = 5pt,
	leftmargin = 0cm,
	rightmargin = 0cm,
	innerbottommargin = 5pt,
	backgroundcolor = dcblue!10	
	]{iBox}

%---------------------------------------------------------------------
%	NEW ENVIRONMENTS
%---------------------------------------------------------------------

%\newenvironment{<name>}{<begin code>}{<end code>} defines a new environment called <name>. At \begin{<name>} the code <begin code> will be executed and at \end{<name>} the <end code> is inserted.

% Creates an environment for each type of theorem and assigns it a theorem text style from the "Theorem Styles" section above and a colored box from above

\newenvironment{definitionE}{\begin{dBox}\begin{definitionT}}{\end{definitionT}\end{dBox}}
\newenvironment{functionE}{\begin{dBox}\begin{functionT}}{\end{functionT}\end{dBox}}
\newenvironment{excerptE}{\begin{dBox}\begin{excerptT}}{\end{excerptT}\end{dBox}}
\newenvironment{instructionE}{\begin{iBox}\begin{instructionT}}{\end{instructionT}\end{iBox}}






